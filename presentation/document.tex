\RequirePackage[l2tabu,orthodox]{nag}
\documentclass{beamer}
\usepackage{tabu}
\usetheme{PaloAlto}
\usecolortheme{orchid}

\title[Lisp Introduction]{Introduction to Lisp}
\author{Jordan Biondo \\ Sean Fisk}
\institute{Grand Valley State University}
\date{\today}
\begin{document}

\begin{frame}
\titlepage
\end{frame}


\begin{frame}{Introduction}
  \begin{itemize}
  \item \textbf{Lis}t \textbf{P}rocessing, also know as Lisp
  \item Second oldest high-level programming language
  \item Invented by John McCarthy at MIT in 1958
  \item Leading family of functional programming languages
  \end{itemize}
\end{frame}

\begin{frame}{Structure of the Language}
\begin{itemize}
\item Parenthesized prefix notation
\item Data is contained in s-expressions
\item Code is data
\item Everything in Lisp is either an atom or a list.
\end{itemize}

\end{frame}{Atoms}

Represent the most basic data types in Lisp.

Examples:

\begin{tabu}{| X | X |}
  \hline
  Numbers & \texttt{9}, \texttt{12.2}, \texttt{9e10}, \texttt{\#x2f}, \texttt{10/3} \\ \hline
  Strings & \texttt{"Bob"}, \texttt{"Lisp is awesome"} \\ \hline
  Characters & \texttt{\#\textbackslash{}a}, \texttt{\#\textbackslash{}linefeed} \\ \hline
\end{tabu}

\end{document}