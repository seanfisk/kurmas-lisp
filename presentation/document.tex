% -*- coding: utf-8; -*-
\RequirePackage[l2tabu,orthodox]{nag}
\documentclass{beamer}
\usetheme{PaloAlto}
\usecolortheme{orchid}

% Using a modified version of minted that supports inline code. See
% the bottom of this page:
% <https://code.google.com/p/minted/issues/detail?id=15>.
\usepackage{minted}
\newminted{cl}{}
\newmint{cl}{}

\usepackage{tabu}

% \usepackage{csquotes}
\usepackage[backend=biber,style=authoryear]{biblatex}
\addbibresource{citations.bib}
\nocite{*}

\usepackage{graphicx}

\usepackage{hyperref}
\hypersetup{colorlinks=true,urlcolor=blue}

\newcommand{\mailtohref}[1]{\href{mailto:#1}{$<$\nolinkurl{#1}$>$}}
\newcommand{\blue}[1]{\textcolor{blue}{#1}}

\title[Lisp Introduction]{Introduction to Lisp}
\author{Jordan Biondo \\ Sean Fisk}
\institute{Grand Valley State University}
\date{\today}

\begin{document}
% In Emacs, you can use `C-x [' and `C-x ]' to jump forward and
% backward slides.

\begin{frame}
  \titlepage
\end{frame}

\begin{frame}{Introduction}
  \begin{itemize}
  \item \blue{Lis}t \blue{P}rocessing, also know as Lisp
  \item Second oldest currently used high-level programming language
  \item Invented by John McCarthy at MIT in 1958
  \item Leading family of functional programming languages
  \end{itemize}
\end{frame}

\begin{frame}{Functional Programming}
  \begin{itemize}
  \item Functional languages view computation as the evaluation of mathematical functions.
  \item Functional programming is based on lambda calculus.
  \item Functions have no side effects:
    \begin{itemize}
    \item They avoid mutable data, i.e., changing values outside of a
      function's scope.
    \item Lisp can be written functionally, but is not a purely
      functional language. It may also be written with typical
      imperative or object-oriented approaches.
    \end{itemize}
  \end{itemize}
\end{frame}

\begin{frame}{Structure of the Language}
  \begin{itemize}
  \item Parenthesized prefix notation
  \item Data is contained in S-expressions
  \item Code is data
  \item Everything in Lisp is either an atom or a list.
  \end{itemize}
\end{frame}

\begin{frame}{Atoms}
  Represent the most basic data types in Lisp.

  Examples:

  \begin{tabu}{| X | X |}
    \hline
    Numbers & \texttt{9}, \texttt{12.2}, \texttt{9e10}, \texttt{\textbackslash\#x2f}, \texttt{10/3} \\ \hline
    Strings & \texttt{"Bob"}, \texttt{"Lisp is awesome"} \\ \hline
    % Couldn't figure out how to get backslashes followed by text
    % syntax-highlighted properly. Oh well.
    Characters & \texttt{\#\textbackslash{}a}, \texttt{\#\textbackslash{}linefeed} \\ \hline
  \end{tabu}
\end{frame}

\begin{frame}{Cons Cells}
  \begin{itemize}
  \item Stands for ``construct''
  \item Data structure containing two pointers
  \item Like a linked-list cell with two elements
    \begin{itemize}
    \item A pointer to the cells value
    \item A pointer to the next cell
    \end{itemize}
  \end{itemize}

  Creating a cons cell: \cl|(cons 1 2)|

  \includegraphics{dot/cons}
\end{frame}

\begin{frame}{\texttt{car} and \texttt{cdr}}
  \texttt{car}:
  \begin{itemize}
  \item \cl|car| returns the value of the first element of a cons cell
  \item \cl|(car (cons 1 2)) -> 1|
  \item Alternate notation: \cl|(first (cons 1 2))|
  \item \cl|car| stands for ``Contents of the Address part of Register''
  \end{itemize}
  \texttt{cdr}:
  \begin{itemize}
  \item \cl|cdr| returns the value of the second element of a cons cell
  \item \cl|(cdr (cons 1 2)) -> 2|
  \item Alternate notation: \cl|(rest (cons 1 2))|
  \item \cl|cdr| stands for ``Contents of the Decrement part of Register''
  \end{itemize}
  The names are historical and do not have any current meaning.
\end{frame}

% `fragile' is needed for the minted block.
\begin{frame}[fragile]{Lists}
\begin{itemize}
\item Ordered collection of cons cells.
\item The \cl|cdr| of each cons is a pointer to the next cons, just like a linked list.
\item The last element in a list has a \cl|nil| \cl|cdr|, signifying the end of the list.
\item Nested lists are expressed in a parenthesized notation known as an \textbf{S-expression}.
\item S-expressions can be though of as trees of cons cells.
\end{itemize}
Example:
\begin{minted}{cl}
((kurmas wolffe engelsma nandigam)
 (c ruby lisp ada))
\end{minted}
\end{frame}

\begin{frame}[fragile]{More List Examples}
  \begin{clcode}
    (A B C)
  \end{clcode}
\includegraphics[height=\textheight/3]{dot/abc}
\begin{clcode}
  (A (B C) D)
\end{clcode}
\includegraphics[height=\textheight/3]{dot/abcd}
\end{frame}

\begin{frame}{Calling Functions}
\begin{itemize}
\item The first element of an evaluated list is the function name to be called.
\item The rest of the elements are arguments to the function.
\item The arguments may themselves contain lists to be evaluated.
\end{itemize}
Examples:
\begin{itemize}
\item \texttt{2 + 3 * 5} would be written \cl|(+ 2 (* 3 5))|
\item \texttt{a and (b or c)} would be written \cl|(and a (or b c))|
\item \texttt{foo(x, y)} (in a C-like language) would be written \cl|(foo x y)|
\end{itemize}
\end{frame}

\begin{frame}{Quoting}
  \begin{itemize}
  \item Quoting delays evaluation of an S-expression.
  \item Without quoting, the first element of a list is treated as a function.
  \end{itemize}
Examples:
\begin{itemize}
\item \cl|(+ 2 3) -> 5|
\item \cl|'(+ 2 3) -> (+ 2 3)|
\end{itemize}
\end{frame}

\begin{frame}[fragile]{Defining Functions}
\blue{De}fine \blue{fun}ctions with \cl|defun|. The general form is:
\begin{clcode}
(defun FUNCTION-NAME (ARGUMENTS...)
  "OPTIONAL-DOCUMENTATION..."
  BODY...)
\end{clcode}
Optional and default parameters are also possible, but we won't go over those today.

Example:
\begin{clcode}
(defun add (first second)
  "Add FIRST to SECOND and return the result."
  (+ first second))
\end{clcode}
Convention dictates that variable names be ALL CAPS in the docstring.
\end{frame}

\begin{frame}[fragile]{Conditionals}
  The general form is:
  \begin{clcode}
    (if cond then else...)
  \end{clcode}
  \begin{itemize}
  \item \texttt{then} is evaluated if \texttt{cond} is not \cl|nil|.
  \item \texttt{else...} is evaluated if \texttt{cond} is \cl|nil|.
  \item \cl|if| returns the value of the expression that was evaluated; either \texttt{then} or \texttt{else...}. This is a general functional programming ``thing''.
  \item To evaluate multiple expressions in \texttt{then}, use \cl|progn| or \cl|block|.
  \item \cl|nil| is false as in other languages, and everything else is true (\cl|t|).
  \item Unlike C, \cl|0| is not considered false.
  \end{itemize}
\end{frame}

\begin{frame}[fragile]{\texttt{progn}}
  \cl|progn| evaluates all expressions in its body and returns the result of the last one. The general form is:
  \begin{clcode}
    (progn body...)
  \end{clcode}
A contrived example of the use of \cl|progn| with \cl|if|:
\begin{clcode}
(if (player-won)
  (progn
    (record-winner player)
    (1+ wins))
  wins)
\end{clcode}
\end{frame}

\begin{frame}[fragile]{Defining Variables}
\texttt{let}:
\begin{itemize}
\item Allows definition of multiple variables that are accessible
  within the scope of the body. The general form is:
\begin{clcode}
(let ((var1 value) (var2 value) ... (varn value)) body...)
\end{clcode}
\item The last statement in the body is returned.
\end{itemize}
\texttt{defvar}:
\begin{itemize}
\item Defines a global variable. The general form is:
  \begin{clcode}
    (defvar name init-value)
  \end{clcode}
\item These variables are global! So don't use this often.
\item Does not fit the functional paradigm.
\end{itemize}
\end{frame}
\begin{frame}[fragile]{Practice Problems: \texttt{my-square}}
  Define a function \texttt{my-square} that returns the square of the argument.

  Here is a skeleton:
  \begin{clcode}
(defun my-square (num)
  "Return the square of NUM."
  ;; Works only for 5 and -5 :)
  25)
  \end{clcode}
\end{frame}

\begin{frame}[fragile]{Practice Problems: \texttt{my-square}: Solution}
  Possible solution:
  \begin{clcode}
(defun my-square (num)
  "Return the square of NUM."
  (* x x))
  \end{clcode}
\end{frame}

\begin{frame}[fragile]{Practice Problems: \texttt{my-count}}
  Define a recursive function that returns the number of elements in a list. You can't use \cl|(length)|!.

Here is a skeleton:
\begin{clcode}
(defun my-count (list)
  "Return the number of elements in LIST."
  ;; Works only for lists of length 10 :)
  10)
\end{clcode}

Expected results:
\begin{clcode}
  (my-count nil) -> 0
  (my-count '()) -> 0
  (my-count '(a b c)) -> 3
  (my-count '(i get the point)) -> 4
\end{clcode}
\end{frame}

\begin{frame}[fragile]{Practice Problems: \texttt{my-count}: Solution}
Possible solution:
  \begin{clcode}
(defun my-count (list)
  "Return the number of elements in LIST."
  (let ((next-cell (cdr l)))
    (if next-cell
        (1+ (my-count next-cell))
      1)))
  \end{clcode}
\end{frame}

\begin{frame}[fragile]{Practice Problems: \texttt{my-last}}
Define a recursive function \texttt{my-last} that returns the last element in a list. You can't use \cl|(last)|! Don't forget the docstring.

Expected results:
\begin{clcode}
  (my-last nil) -> nil
  (my-last '(hello world)) -> (world)
  (my-last '(a b c d)) -> (d)
\end{clcode}
\end{frame}

\begin{frame}[fragile]{Practice Problems: \texttt{my-last}: Solution}
  Possible solution:
  \begin{clcode}
(defun my-last (list)
  "Return the last cons cell of LIST.
Should behave similar to `last'."
  (let ((next-cell (cdr list)))
    (if next-cell
        (my-last next-cell)
      list))
  \end{clcode}
\end{frame}

\begin{frame}[fragile]{Practice Problems: \texttt{my-reverse}}
  Define a recursive function \texttt{my-reverse} that returns a copy
  of the list in reverse order. You can't use \cl|(reverse)|!

Expected results:
\begin{clcode}
  (my-reverse nil) -> nil
  (my-reverse '(a)) -> (a)
  (my-reverse '(a b c d)) -> (d c b a)
\end{clcode}
\end{frame}

\begin{frame}[fragile]{Practice Problems: \texttt{my-reverse}: Solution}
  Possible solution:
  \begin{clcode}
(defun my-reverse (list)
  "Return a reversed copy of LIST."
  (if (cdr list)
      (append (last list) (my-reverse (butlast list)))
    list))
  \end{clcode}
\end{frame}

\begin{frame}{Do people actually use Lisp?}
  \textbf{Yes!}
  \begin{itemize}
  \item \textbf{AutoLISP:} The integrated scripting language for AutoCAD and
    other Autodesk products.
  \item \textbf{Mirai:} A 3D graphics suite used to animate Gollum's face in The Lord of the Rings.
  \item \textbf{Dynamic Analysis and Replanning Tool (DART):} An AI research
    project part of the United States Defense Advanced Research
    Projects Agency (DARPA).
  \item \textbf{The Emacs Editor:} 80\% is written in Emacs Lisp,
    adding up to an insane 1,131,162 lines of Emacs Lisp code!
  \end{itemize}
\end{frame}

\begin{frame}{Credits}
  Written by:
  \begin{itemize}
  \item Sean Fisk \mailtohref{fiskse@mail.gvsu.edu}
  \item Jordon Biondo \mailtohref{biondoj@mail.gvsu.edu}
  \end{itemize}
  The following free and open-source software was used to produce this presentation: \\
  \LaTeX, Beamer, Biber, Biblatex, Minted, Pygments, Emacs, AUCTeX, SCons, Clisp, cloc, Graphviz
\end{frame}
\begin{frame}{References}
  \printbibliography
\end{frame}
\end{document}