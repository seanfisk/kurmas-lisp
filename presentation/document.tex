% -*- coding: utf-8; -*-
\RequirePackage[l2tabu,orthodox]{nag}
\documentclass{beamer}
\usepackage{minted}
\newminted{cl}{}
\newmint{cl}{}
\usepackage{tabu}
\usetheme{PaloAlto}
\usecolortheme{orchid}

\title[Lisp Introduction]{Introduction to Lisp}
\author{Jordan Biondo \\ Sean Fisk}
\institute{Grand Valley State University}
\date{\today}

\begin{document}

\begin{frame}
\titlepage
\end{frame}


\begin{frame}{Introduction}
  \begin{itemize}
  \item \textbf{Lis}t \textbf{P}rocessing, also know as Lisp
  \item Second oldest high-level programming language
  \item Invented by John McCarthy at MIT in 1958
  \item Leading family of functional programming languages
  \end{itemize}
\end{frame}

\begin{frame}{Structure of the Language}
\begin{itemize}
\item Parenthesized prefix notation
\item Data is contained in s-expressions
\item Code is data
\item Everything in Lisp is either an atom or a list.
\end{itemize}

\end{frame}

\begin{frame}{Atoms}
Represent the most basic data types in Lisp.

Examples:

\begin{tabu}{| X | X |}
  \hline
  Numbers & \texttt{9}, \texttt{12.2}, \texttt{9e10}, \texttt{\#x2f}, \texttt{10/3} \\ \hline
  Strings & \texttt{"Bob"}, \texttt{"Lisp is awesome"} \\ \hline
  Characters & \texttt{\#\textbackslash{}a}, \texttt{\#\textbackslash{}linefeed} \\ \hline
\end{tabu}
\end{frame}

\begin{frame}{Cons Cells}
\begin{itemize}
\item Stands for ``construct''
\item Data structure containing two pointers
\item Like a linked-list cell with two elements
  \begin{itemize}
  \item A pointer to the cells value
  \item A pointer to the next cell
  \end{itemize}
\end{itemize}

Creating a cons cell: \cl|(cons 1 2)|
\end{frame}

\begin{frame}{car and cdr}
  \begin{itemize}
  \item \texttt{car} returns the value of the first element of a cons cell
  \item \cl|(car (cons 1 2)) -> 1|
  \item Alternate notation: \cl|(first (cons 1 2))|
  \item \texttt{cdr} returns the value of the second element of a cons cell
  \item \cl|(cdr (cons 1 2)) -> 2|
  \item Alternate notation: \cl|(rest (cons 1 2))|
  \item \texttt{car} stands for ``Contents of the Address part of Register number''
  \item \texttt{cdr} stands for ``Contents of the Decrement part of Register number''
  \item The names are historical and do not have any current meaning.
  \end{itemize}
\end{frame}

\end{document}