% -*- coding: utf-8; -*-
\RequirePackage[l2tabu,orthodox]{nag}
\documentclass{beamer}
\usetheme{PaloAlto}
\usecolortheme{orchid}

% Using a modified version of minted that supports inline code. See
% the bottom of this page:
% <https://code.google.com/p/minted/issues/detail?id=15>.
\usepackage{minted}
\newminted{cl}{}
\newmint{cl}{}

\usepackage{tabu}

\newcommand{\blue}[1]{\textcolor{blue}{#1}}

\title[Lisp Introduction]{Introduction to Lisp}
\author{Jordan Biondo \\ Sean Fisk}
\institute{Grand Valley State University}
\date{\today}

\begin{document}
% In Emacs, you can use `C-x [' and `C-x ]' to jump forward and
% backward slides.

\begin{frame}
  \titlepage
\end{frame}

\begin{frame}{Introduction}
  \begin{itemize}
  \item \blue{Lis}t \blue{P}rocessing, also know as Lisp
  \item Second oldest currently used high-level programming language
  \item Invented by John McCarthy at MIT in 1958
  \item Leading family of functional programming languages
  \end{itemize}
\end{frame}

\begin{frame}{Functional Programming}
  \begin{itemize}
  \item Functional languages view computation as the evaluation of mathematical functions.
  \item Functional programming is based on lambda calculus.
  \item Functions have no side effects:
    \begin{itemize}
    \item They avoid mutable data, i.e., changing values outside of a
      function's scope.
    \item Lisp can be written functionally, but is not a purely
      functional language. It may also be written with typical
      imperative or object-oriented approaches.
    \end{itemize}
  \end{itemize}
\end{frame}

\begin{frame}{Structure of the Language}
  \begin{itemize}
  \item Parenthesized prefix notation
  \item Data is contained in S-expressions
  \item Code is data
  \item Everything in Lisp is either an atom or a list.
  \end{itemize}
\end{frame}

\begin{frame}{Atoms}
  Represent the most basic data types in Lisp.

  Examples:

  \begin{tabu}{| X | X |}
    \hline
    Numbers & \cl|9|, \cl|12.2|, \cl|9e10|, \cl|\#x2f|, \cl|10/3| \\ \hline
    Strings & \cl|"Bob"|, \cl|"Lisp is awesome"| \\ \hline
    % Couldn't figure out how to get backslashes followed by text
    % syntax-highlighted properly. Oh well.
    Characters & \texttt{\#\textbackslash{}a}, \texttt{\#\textbackslash{}linefeed} \\ \hline
  \end{tabu}
\end{frame}

\begin{frame}{Cons Cells}
  \begin{itemize}
  \item Stands for ``construct''
  \item Data structure containing two pointers
  \item Like a linked-list cell with two elements
    \begin{itemize}
    \item A pointer to the cells value
    \item A pointer to the next cell
    \end{itemize}
  \end{itemize}

  Creating a cons cell: \cl|(cons 1 2)|
\end{frame}

\begin{frame}{car and cdr}
  car:
  \begin{itemize}
  \item \cl|car| returns the value of the first element of a cons cell
  \item \cl|(car (cons 1 2)) -> 1|
  \item Alternate notation: \cl|(first (cons 1 2))|
  \item \cl|car| stands for ``Contents of the Address part of Register''
  \end{itemize}
  cdr:
  \begin{itemize}
  \item \cl|cdr| returns the value of the second element of a cons cell
  \item \cl|(cdr (cons 1 2)) -> 2|
  \item Alternate notation: \cl|(rest (cons 1 2))|
  \item \cl|cdr| stands for ``Contents of the Decrement part of Register''
  \end{itemize}
  The names are historical and do not have any current meaning.
\end{frame}

% `fragile' is needed for the minted block.
\begin{frame}[fragile]{Lists}
\begin{itemize}
\item Ordered collection of cons cells.
\item The \cl|cdr| of each cons is a pointer to the next cons, just like a linked list.
\item The last element in a list has a \cl|nil| \cl|cdr|, signifying the end of the list.
\item Nested lists are expressed in a parenthesized notation known as an \textbf{S-expression}.
\item S-expressions can be though of as trees of cons cells.
\end{itemize}
Example:
\begin{minted}{cl}
((kurmas wolffe engelsma nandigam)
 (c ruby lisp ada))
\end{minted}
\end{frame}

\begin{frame}{Calling functions}
\begin{itemize}
\item The first element of an evaluated list is the function name to be called.
\item The rest of the elements are arguments to the function.
\item The arguments may themselves contain lists to be evaluated.
\end{itemize}
Examples:
\begin{itemize}
\item \texttt{2 + 3 * 5} would be written \cl|(+ 2 (* 3 5))|
\item \texttt{a and (b or c)} would be written \cl|(and a (or b c))|
\item \texttt{foo(x, y)} (in a C-like language) would be written \cl|(foo x y)|
\end{itemize}
\end{frame}
\end{document}